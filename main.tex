\documentclass{beamer}
\usepackage{listings}
\usetheme{Madrid}
\usecolortheme{default}

\title{Formal Methods}
\author{Rodolfo Lopez}
\institute{USD}
\date[March 2023]{March 2023}

\begin{document}

\begin{frame}
\titlepage
\end{frame}

\begin{frame}{Overview}
\tableofcontents
\end{frame}

\section{What are Formal Methods?}

\begin{frame}{Definition}
\begin{itemize}
\item Formal methods are mathematical techniques used for the specification, verification, and validation of software and hardware systems.
\item Formal methods can be used to ensure the correctness, reliability, and safety of critical systems, such as those used in aerospace, defense, transportation, and healthcare.
\end{itemize}
\end{frame}

\section{Model Checking}

\begin{frame}{Model Checking}
\begin{itemize}
\item Model checking is a formal verification technique that checks whether a model of a system satisfies a given specification.
\item Let $M$ be a model of a system and $\phi$ be a specification of the system. If we can check that $M$ satisfies $\phi$ using model checking, then we can guarantee that the system satisfies $\phi$ as well.
\item Model checking can be used to verify that a circuit design meets its functional requirements and does not have any logical errors that could cause it to malfunction.
\end{itemize}
\end{frame}

\begin{frame}{Model Checking Example}
\begin{itemize}
\item Suppose $M$ is a model of a system and $\phi$ is a specification of the system.
\item To check if $M$ satisfies $\phi$, we construct the negation of $\phi$, denoted as $\neg \phi$.
\item We then check if there exists a state in $M$ where $\neg \phi$ holds. If such a state exists, then $M$ does not satisfy $\phi$. Otherwise, $M$ satisfies $\phi$.
\item This proof is based on the soundness and completeness of propositional logic, which ensures that the negation of a true proposition is false and vice versa.
\end{itemize}
\end{frame}

\section{Lean Proof Assistant}

\begin{frame}{Lean Proof Assistant}
\begin{itemize}
\item Lean is a powerful proof assistant that allows users to write and verify mathematical proofs using formal logic.
\item It is based on dependent type theory, which allows for the definition of complex data structures and logical propositions.
\item Lean provides a user-friendly interface for writing and checking proofs, making it an ideal tool for formal methods.
\end{itemize}
\end{frame}

\begin{frame}[fragile]{Lean Example Code}
\begin{lstlisting}
inductive mynat
| zero : mynat
| succ (n : mynat) : mynat
\end{lstlisting}
\begin{lstlisting}
lemma zero_add (n : mynat) : zero + n = n :=
begin
induction n with d hd,
{
  rw add_zero,
},
{
  rw add_succ,
  rw hd,
}
end
\end{lstlisting}
\end{frame}

\section{Modern Applications}

\begin{frame}{Modern Applications}
Formal methods and proof assistants like Lean are used in a variety of applications, including:
\begin{itemize}
\item \textbf{Software Verification}: Formal methods can be used to verify that software is correct and free of bugs. This is particularly important in safety-critical systems, such as medical devices and transportation systems.
\item \textbf{Hardware Verification}: Formal methods can also be used to verify that hardware designs are correct and meet specifications.
\item \textbf{Artificial Intelligence}: Formal methods are increasingly being used in the development of AI systems, to ensure that they are safe, reliable, and free of bias.
\item \textbf{Blockchain Technology}: Formal methods are used to verify the correctness and security of smart contracts and other blockchain-based systems.
\end{itemize}
\end{frame}

\section{Challenges}

\begin{frame}{Challenges}
\begin{itemize}
\item Formal methods require a high level of mathematical expertise and are often time-consuming and expensive to apply.
\item Formal methods can only guarantee correctness with respect to a given specification, which may not capture all possible scenarios or requirements of a system.
\end{itemize}
\end{frame}

\begin{frame}{Conclusion}
\begin{itemize}
\item Formal methods and proof assistants like Lean are powerful tools for specifying, developing and verifying software and hardware systems.
\item They are used in a variety of applications and are particularly important in safety-critical systems, AI, and blockchain technology.
\end{itemize}
\end{frame}

\end{document}





